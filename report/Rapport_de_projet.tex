\documentclass[12pt, a4paper]{article}

\usepackage[utf8]{inputenc}
\usepackage[T1]{fontenc}
\usepackage[francais]{babel}
\usepackage{graphicx}
\usepackage{fontawesome}
\usepackage{hyperref}
\usepackage{fullpage}
\usepackage{caption}
\usepackage{subcaption}
\usepackage{enumitem}

\usepackage{helvet}
\renewcommand{\familydefault}{\sfdefault}


\begin{document}


%%%%%%%%%%%%%%%%%%%%%%%%%%%%%%%%%%%%%%%%%%%%%%%%%%%%%
%Couverture:
%%%%%%%%%%%%%%%%%%%%%%%%%%%%%%%%%%%%%%%%%%%%%%%%%%%%%
\begin{titlepage}
  \begin{center}

    % Title
    \rule{\linewidth}{0.5mm} \\[0.4cm]
    { \huge \bfseries {\LARGE{Rapport de projet}}
    \\Retouche d'images sur Android\\[0.4cm] }
    \rule{\linewidth}{0.5mm} \\[1.5cm]
    
    {\Large {
      Manuel Ricardo Guevara Garban\\
      Loïc Lachiver\\
      Romain Pigret-Cadou\\
      Sofiane Menadjlia
    }}\\[1.5cm]
    {\LARGE L3 Informatique}\\[0.5cm]
    
    {\Large Projet technologique}\\[0.5cm]
    {\Large 2020}\\[1.5cm]
    
    \includegraphics[width=1\textwidth]{report_src/logoUB.jpg}
    
    
  \end{center}
\end{titlepage}



\tableofcontents
\clearpage 


%%%%%%%%%%%%%%%%%%%%%%%%%%%%%%%%%%%%%%%%%%%%%%%%%%%%%
% Introduction :
%%%%%%%%%%%%%%%%%%%%%%%%%%%%%%%%%%%%%%%%%%%%%%%%%%%%%
\section{Introduction :} \label{intro}
Ce rapport synthétise notre travail de développement d'une application de retouche d'image exécutable sur la plateforme mobile Android.
\\
Réalisé dans le cadre de l'UE \textit{Projet technologique} lors du 2ème semestre de Licence 3 informatique à l'Université de Bordeaux, les objectifs de ce rendu de groupe étaient les suivant:
\begin{itemize} [label=\textbullet]
  \item Réaliser une application graphique Android
  \item Permettre de charger, éditer et sauvegarder facilement une image
  \item Proposer des effets par simple modification de pixels
  \item Proposer des effets manipulant des histogrammes
  \item Proposer des effets de convolution
  \item Utiliser la technologie d'accélération RenderScript
  \item Gérer le développement d'un projet de groupe
\end{itemize}

\vspace{1cm}
\faArrowRight Lien GitHub du projet :
\href{https://github.com/picachoc/pimp-android}{https://github.com/picachoc/pimp-android}
\vspace{1cm}

\clearpage

%%%%%%%%%%%%%%%%%%%%%%%%%%%%%%%%%%%%%%%%%%%%%%%%%%%%%
%L'application :
%%%%%%%%%%%%%%%%%%%%%%%%%%%%%%%%%%%%%%%%%%%%%%%%%%%%%
\section{L'application :}
\textbf{Version minimale Android requise :} Oreo (8.0) (API level 26)

\subsection{Lancement :}
Le projet n'a pas encore été conçu pour être disponible sous un format APK.
\\
Le dépôt git est conçu pour être importé dans Android Studio.
\\
Après avoir cloné le dépôt, rendez vous sur Android Studio puis 
{\textbf {File > New > Import Project}}
et sélectionnez le dossier créé par le clone.

\subsection{Utilisation :}
A l'ouverture de l'application vous obtiendrez l'interface suivante :
\begin{center}
    \begin{minipage}{.48\textwidth}
      \includegraphics[width=1\textwidth]{report_src/app_manual/first_activity_preview.JPG}
    \end{minipage}
    \begin{minipage}{.48\textwidth}
        Appuyez sur le premier bouton (\faPhoto) pour ouvrir votre galerie photo et choisir une image à importer.
        \\

        Ou appuyez sur le second bouton (\faCamera) pour ouvrir la caméra de votre appareil Android. Capturez alors une image, elle sera sauvegardée dans votre galerie et instantanément importée dans l'application.
        \\

        Il sera nécessaire à la toute première utilisation de ces fonctionnalités d'autoriser l'application à accéder à la galerie et/ou la caméra.
    \end{minipage}
\end{center}
\clearpage

Après avoir choisi une image vous accéderez à la page d'édition principale :
\begin{center}
    \begin{minipage}{.48\textwidth}
      \includegraphics[width=1\textwidth]{report_src/app_manual/main_activity_preview.JPG}
    \end{minipage}
    \begin{minipage}{.48\textwidth}
        Il vous sera possible d'importer une nouvelle image depuis le menu {\faChevronCircleDown}  en haut à droite :
        \begin{center}
        \includegraphics[width=0.8\textwidth]{report_src/app_manual/3dot_button_preview.JPG}
        \end{center}    

        Le bouton {\faSave} permettra d'exporter votre image dans votre galerie, cette opération peut prendre un peu de temps car l'image exportée sera de même taille que l'image d'origine.
    \end{minipage}
\end{center}

En bas de l'écran vous trouverez la liste des effets disponibles applicables sur votre image, faites défiler de droite à gauche.
\\

Le bouton {\faMailReply} annule tout les changements appliqués à l'image en la restaurant à son état d'origine.
\clearpage

Le bouton {\faInfoCircle} quand à lui affiche les informations suivantes à propos de l'image :
\begin{center}
    \begin{minipage}{.48\textwidth}
      \includegraphics[width=1\textwidth]{report_src/app_manual/info_fragment_preview.JPG}
    \end{minipage}
    \begin{minipage}{.48\textwidth}
        Les sections {\faPhoto} {\faCalendar} {\faChrome} et {\faMapMarker} permettent d'afficher des informations respectivement sur le fichier image, sur sa date de prise de vue, sur l'appareil source et sur ses coordonnées géographiques.
        \\
        Ces sections ne sont pas toujours visibles et dépendent du fichier image.
        \\

        Quand à la section {\faCog} elle donne les dimensions de l'aperçu actuellement affiché dans l'application, il peut être plus petit que l'image d'origine afin de préserver la mémoire du téléphone et le temps d'exécution. L'image exportée en revanche sera aux bonnes dimensions.
    \end{minipage}
\end{center}
\clearpage

Après avoir sélectionné un effet sur la liste en bas de l'écran, des éléments d'interface vont se superposer à votre image :
\begin{center}
    \begin{minipage}{.48\textwidth}
      \includegraphics[width=1\textwidth]{report_src/app_manual/effect_fragment_preview.JPG}
    \end{minipage}
    \begin{minipage}{.48\textwidth}
        Le bouton {\faRemove} en haut à gauche vous fait revenir à la liste d'effets et annule l'effet courant.
        \\

        Le bouton {\faCheck} en haut à droite applique l'effet et vous fait revenir à la liste d'effets.
        \\

        En bas des curseurs ou des boutons peuvent apparaître, selon l'effet sélectionné. Manipulez les pour voir en temps réel l'impact sur l'image.
    \end{minipage}
\end{center}

\vspace{2cm}
\textbf{NB:} L'entièreté de cette application est verrouillée en mode portrait.

\clearpage

%%%%%%%%%%%%%%%%%%%%%%%%%%%%%%%%%%%%%%%%%%%%%%%%%%%%%
% Effets disponibles :
%%%%%%%%%%%%%%%%%%%%%%%%%%%%%%%%%%%%%%%%%%%%%%%%%%%%%
\section{Effets :}
\emph{Interface :}

L'utilisateur choisit un effet parmi la liste d'effets affichée en bas de l'écran, ce qui affiche les 
réglages disponibles (s'il y en a). L'utilisateur peut ensuite choisir de confirmer ou d'annuler la modification
apportée par l'effet choisi.
\\

\emph{Structure du code :}

Les effets sont rassemblés dans le package filters (fr.ubordeaux.pimp.filters).
La classe \emph{Retouching} contient les réglages de luminosité, contraste, saturation et teinte.
La classe \emph{Convolution} contient tous les effets liés à la convolution (flou, détection de contour etc.).
Toutes ces méthodes sont appelées lors de l'appui de boutons ou de glissement de seekbars. Les seekbars ont généralement
une étendue allant de 0 à 255 (sauf pour les réglages de teinte), et certains effets nécessitent des valeurs pouvant être
négatives. Ainsi la valeur de seekbar est modifiée dans la méthode appelante de ces mêmes effets.

Tous les effets nécessitant l'histogramme utilise celui de la \emph{valeur}, soit le maximum entre les trois canaux RGB.

\subsection{Luminosité (Brightness) :}
\includegraphics[width=0.5\textwidth]{report_src/brightness_low.jpeg}
\includegraphics[width=0.5\textwidth]{report_src/brightness_high.jpeg}

\emph{Méthode appelante : Retouching.setBrightness()}

\emph{Script : brightness.rs}
\\

Ce réglage ajoute une valeur (positive ou négative) aux trois canaux RGB de l'image. Cette valeur est fixée par la seekbar.
Les valeurs sont tronquées entre 0 et 255, par conséquent on perd de l'information dans les valeurs extrêmes de luminosité.

Cet effet n'utilise pas la luminosité existante de l'image, ainsi on peut obtenir des résultats qui sont parfois discutables, par exemple
le noir qui s'éclaircit et inversement pour le blanc. Pour pallier à ce problème, on pourrait introduire une multiplication afin de modifier
la luminosité proportionnellement à celle existante. Cependant, cette solution modifie aussi le contraste, nous avons donc choisi
de laisser l'algorithme tel quel.


\subsection{Contraste (Contrast) :}

\emph{Méthode appelante : Retouching.dynamicExtensionRGB()}

\emph{Script : dynamicExtension.rs}
\\

Ce réglage effectue une extension linéaire de dynamique. Les nouveaux extremum de l'histogramme sont définis à partir de la position de la seekbar.
La dynamique est ainsi étendue autour d'une valeur se situant au milieu des deux anciens extremum de l'histogramme*. On a donc une image uniforme lorsque
l'on règle le contraste au minimum. En augmentant le contraste, les extremum peuvent sortir de l'intervalle [0;255], ce qui provoque une distorsion de l'image.
\\

*Il serait peut-être plus judicieux de prendre la médiane de l'histogramme cumulé afin d'avoir une valeur qui représente mieux la "valeur moyenne" de l'image.


\subsection{Saturation (Saturation) :}

\emph{Méthode appelante : Retouching.setSaturation()}

\emph{Script : saturation.rs}
\\

Ce réglage permet de régler la saturation de l'image. Soit S la saturation existante, S' la nouvelle saturation et F le facteur de saturation. S et S' vont de 0 à 1.

On a S' = S + F * (1 - S) * S. 

On observe que la nouvelle saturation est proportionnelle à deux facteurs : l'espace restant avant une saturation totale (1-S) et la saturation existante S.
Par conséquent, en augmentant la saturation, chaque pixel tend vers sa saturation maximale, tout en garantissant une saturation proportionnelle à celle existante, évitant ainsi de saturer le gris.

\subsection{Egalisation d'histogramme (Enhance) :}

\emph{Méthode appelante : Retouching.histogramEqualization()}

\emph{Scripts : cumulativeHistogram.rs, assignLut.rs} 
\\

Comme son nom l'indique, cet effet utilise l'égalisation d'histogramme afin d'améliorer le contraste.
On calcule d'abord l'histogramme cumulé et on en déduit la LUT (Look Up Table), que nous assignons ensuite à chaque pixel.
\\

Cependant, afin d'égaliser l'histogramme, l'algorithme éclaircit les zones sombres, ce qui donne un résultat peu convaincant sur les images de faible luminosité.
Une solution à ce problème est l'égalisation d'histogramme adaptative (CLAHE), mais nous n'avons pas eu le temps de l'implémenter pour ce rendu.


\clearpage

%%%%%%%%%%%%%%%%%%%%%%%%%%%%%%%%%%%%%%%%%%%%%%%%%%%%%
%Structure du projet:
%%%%%%%%%%%%%%%%%%%%%%%%%%%%%%%%%%%%%%%%%%%%%%%%%%%%%
\section{Structure du projet :}

\subsection{Structure graphique Android et navigation :}
\includegraphics[width=1\textwidth]{report_src/app_flowchart_fragments.PNG}

Pour afficher certains éléments d'interface comme par exemple les informations de l'image (bouton \faInfoCircle) nous utilisons des \textbf{Fragment}. En effet une activité supplémentaire n'est pas nécessaire car cette petite partie de l'application ne correspond pas à un point d'entrée de l'application. Par ailleurs changer de fragment (plutôt que de changer directement de layout) pourrait faciliter l'implémentation d'une interface différente, pour tablette par exemple.
De même, la liste d'effets et leurs paramètres respectifs sont aussi contenus dans des fragments séparés. Cela permet de gérer plus simplement leur affichage et de clarifier le code.
\\

On notera que dans la structure actuelle de l'application, l'image actuellement éditée est contenue et manipulée depuis l'activité principale. Les fragments n'apportent à l'application que des éléments d'interface.
\\

Une seconde activité est cependant utilisée pour la page d'accueil à l'ouverture de l'application, cette \textbf{FirstActivity} utilise des méthodes génériques de \textbf{ActivityIO} afin de gérer l'ouverture de la galerie ou de la caméra. L'application reste dans cette activité tant qu'une \textbf{Uri} valable ($\approx$ chemin) n'a pas été sélectionnée. Ensuite cette Uri est transférée à \textbf{MainActivity} qui va alors charger cette première Image, en cas de problème de chargement l'application peut retourner dans FirstActivity.

\subsection{Classe \textbf{Image} :} \label{classeImage}
Cette classe a été conçu comme une alternative à l'utilisation directe de la classe \textbf{Bitmap} fournie par Android.
\\
Le coeur de la classe est évidement une instance de Bitmap, qu'il est possible de récupérer à tout moment. Par ailleurs la classe offre des fonctionnalités supplémentaires, parmi celle ci notamment la possibilité de restaurer l'image à son état au moment de sa création ou de son chargement via la méthode \textbf{reset()}, ou d'annuler les dernières modifications apportées par un effet grâce aux méthodes \textbf{quicksave()} et \textbf{discard()}.
\\

On notera la nécessité pour Image d'avoir la référence d'une Activité de l'application, en effet elle est requise à plusieurs moments par les librairies Android pour charger la Bitmap en mémoire.

\subsubsection{Classe \textbf{ImageInfo} :}
La classe Image \ref{classeImage} génère et garde une instance de la classe \textbf{ImageInfo}, cette classe contient un grand nombre de valeurs à propos de l'Image (dimensions, coordonnées GPS, date de prise de vue, ...).
\\
L'idée de cette classe était d'empaqueter toutes ces informations afin de faciliter le passage de ces informations à travers des Fragments ou des Activités (voir \ref{navig}). On notera que tous les accesseurs appliquent des opérations de formatage sur ces données, certaines opérations pourraient être déplacées dans les constructeurs si elles venaient à être utilisées régulièrement.

\subsection{Package \textbf{util} :}
Ce package contient de nombreuses classes contenant des méthodes statiques permettant une meilleur factorisation du code.
\\
La classe \textbf{Utils} offre par exemple des méthodes pour récupérer la taille de l'écran ou pour calculer un ratio de redimensionnement.
\\
La classe \textbf{BitmapIO} permet d'effectuer le chargement d'une Bitmap de plusieurs manières, depuis les resources ou un autre emplacement du téléphone, et avec la taille voulue.
\\
La classe \textbf{Effects} est une énumération des effets disponibles. Les enums sont passés en argument du chargement du fragment EffectSettings, afin d'afficher les bon paramètres d'effets.
\\
La classe \textbf{Kernels} contient tous les kernels de convolution (voir \ref{kernels}).
\clearpage

%%%%%%%%%%%%%%%%%%%%%%%%%%%%%%%%%%%%%%%%%%%%%%%%%%%%%
% Tests de performance :
%%%%%%%%%%%%%%%%%%%%%%%%%%%%%%%%%%%%%%%%%%%%%%%%%%%%%
\section{Performances :}
Tous les tests de performance présentés dans cette section ont été effectués sur un Nokia 3.1.
Voici un résumé de ses caractéristiques :
\begin{itemize} [label=\textbullet]
  \item \textbf{Version} : Android 9
  \item \textbf{Résolution}	: 1440 x 720 pixels
  \item \textbf{Cadence processeur} : 4 coeurs 1.5 GHz
  \item \textbf{RAM} : 2 Go
\end{itemize}

\subsection{Temps d'exécution :}

Les temps d'exécution présentés ne tiennent pas compte du premier temps de chargement du script renderscript.
\\

Voici les temps d'exécution \textbf{sur 10 appels} de chaque effet pour une image de dimension \textbf{425x265px}.
\\

\begin{tabular}{||l||c|c||c|c||}
    \hline
    \hline
    \textbf{Effet} & \textbf{Temps d'exécution en ms} (min | max | moyenne | écart-type)\\
    \hline
    \hline
    Brightness & 12.0 | 46.0 | 16.9 | 9.73\\
    \hline
    Contrast & 16.0 | 28.0 | 20.5 | 3.80\\
    \hline
    Saturation & 13.0 | 23.0 | 15.2 | 2.929\\
    \hline
    Enhance & 22.0 | 32.0 | 26.4 | 3.00\\
    \hline
    To gray & 9.0 | 11.0 | 10.0 | 0.63\\
    \hline
    Invert & 8.0 | 10.0 | 8.4 | 0.66\\
    \hline
    Change hue & 11.0 | 18.0 | 13.4 | 1.90\\
    \hline
    Keep hue & 10.0 | 11.0 | 10.6 | 0.48\\
    \hline
    Gaussian blur 3x3 & 17.0 | 21.0 | 18.8 | 1.24\\
    \hline
    Gaussian blur 25x25 & 47.0 | 51.0 | 48.5 | 1.36\\
    \hline
    Mean blur 3x3 & 16.0 | 19.0 | 18.2 | 1.07\\
    \hline
    Mean blur 25x25 & 54.0 | 60.0 | 55.8 | 1.72\\
    \hline
    Sharpen 3x3 & 21.0 | 25.0 | 23.6 | 1.11\\
    \hline
    Sharpen 13x13 & 136.0 | 165.0 | 146.3 | 8.97\\
    \hline
    Sobel filter & 21.0 | 31.0 | 24.7 | 2.53\\
    \hline
    Prewitt filter & 24.0 | 27.0 | 25.3 | 1.09\\
    \hline
    Laplacian filter & 22.0 | 26.0 | 23.6 | 1.11\\
    \hline
    \hline
  \end{tabular}
\\
\newpage
Voici les temps d'exécution \textbf{sur 10 appels} de chaque effet pour une image de dimension \textbf{3400x2118px}.
\\

\begin{tabular}{||l||c|c||c|c||}
    \hline
    \hline
    \textbf{Effet} & \textbf{Temps d'exécution en ms} (min | max | moyenne | écart-type)\\
    \hline
    \hline
    Brightness & 105.0 | 311.0 | 145.8 | 56.95\\
    \hline
    Contrast & 163.0 | 202.0 | 176.9 | 12.73\\
    \hline
    Saturation & 209.0 | 236.0 | 221.4 | 8.34\\
    \hline
    Enhance & 256.0 | 268.0 | 261.3 | 3.74\\
    \hline
    To gray & 125.0 | 138.0 | 129.6 | 3.35\\
    \hline
    Invert & 84.0 | 105.0 | 94.7 | 8.69\\
    \hline
    Change hue & 218.0 | 319.0 | 231.7 | 29.48\\
    \hline
    Keep hue & 162.0 | 181.0 | 168.4 | 5.06\\
    \hline
    Gaussian blur 3x3 & 245.0 | 265.0 | 256.4 | 6.63\\
    \hline
    Gaussian blur 25x25 & 1453.0 | 1484.0 | 1462.2 | 9.08\\
    \hline
    Mean blur 3x3 & 245.0 | 652.0 | 358.5 | 146.25\\
    \hline
    Mean blur 25x25 & 1461.0 | 2447.0 | 1716.6 | 318.34\\
    \hline
    Sharpen 3x3 & 433.0 | 524.0 | 471.7 | 25.25\\
    \hline
    Sharpen 13x13 & 5112.0 | 8974.0 | 5679.2 | 1119.49\\
    \hline
    Sobel filter & 450.0 | 497.0 | 469.4 | 14.47\\
    \hline
    Prewitt filter & 452.0 | 490.0 | 472.5 | 10.39\\
    \hline
    Laplacian filter & 411.0 | 533.0 | 460.4 | 40.08\\
    \hline
    \hline
  \end{tabular}

\subsubsection*{Conclusion sur les temps d'exécution}

Mesurer les temps d'exécution nous a permis de nous rendre compte du gain de temps de la convolution séparable. Le flou gaussien (implémenté en séparable) 25x25 prend 1462.2 ms, tandis que le sharpen 13x13 (non séparable) prend 5679.2 ms.  

Sharpen 3x3, Sobel, Prewitt et Laplacian ont le même temps d'exécution car elles correspondent toutes à des convolutions classique avec un kernel 3x3.

De plus, on remarque que l'effet "Invert" est le plus rapide. Cela s'explique par le fait que c'est le seul effet qui manipule directement des entiers. Peut-être pourrions-nous appliquer cette méthode à d'autres effets tels que la convolution, afin d'optimiser les temps d'exécution.

\subsection{Mémoire :}

Voici l'utilisation de la mémoire vive du téléphone au cours de l'utilisation de l'application. On démarre le benchmarking à partir du chargement de l'image (après fistActivity), puis on applique tous les effets disponibles en annulant à chaque fois.
\subsubsection*{Image de 425x265px}
\includegraphics[width=1\textwidth]{report_src/memory/425x265.PNG}

\subsubsection*{Image de 3400x2118px}
\includegraphics[width=1\textwidth]{report_src/memory/3400x2118.PNG}

\subsubsection*{Conclusion sur l'utilisation de la mémoire}
On observe que la mémoire est plutôt bien gérée et qu'elle reste constante tout au long de l'utilisation. De plus, le garbage collector vient régulièrement libérer la mémoire allouée.

En analysant le tas de l'application ("app heap" sur le profiler), on se rend compte que la sauvegarde de l'image (effectuée avec la méthode quicksave()) y occupe la plus grande place. La sauvegarde est stockée sous forme de tableau d'entiers, on peut donc difficilement le rendre moins lourd en mémoire. En revanche, la file d'effets (\textbf{\ref{file_effets}}) pourrait nous permettre d'annuler les modifications sans avoir à stocker une sauvegarde. En effet, si l'utilisateur a appliqué n effets et qu'il souhaite annuler la dernière modification, on peut ré-appliquer tous les effets jusqu'à n-1 sur l'image d'origine. Cependant, on perdrait évidemment en temps.

\newpage

\subsubsection*{Bug détecté}

Le profiling de la mémoire nous a permis de détecter un problème que l'on a résolu. En effet, on remarquait une augmentation constante de la mémoire lorsque l'on appliquait un flou :
\\

\includegraphics[width=1\textwidth]{report_src/memory/3400x2118_with_bug.PNG}
\\

C'était du à notre méthode convolve2dSeparable. La convolution séparée nécessite une allocation temporaire pour stocker le résultat de la première convolution (horizontale). 
Elle était créée puis libérée directement depuis notre script convolution.rs (méthode convolutionSeparable). Cependant, la libération ne fonctionnait apparemment pas, nous avons donc géré l'allocation de ce bitmap temporaire depuis java, ce qui a réglé le problème.

\clearpage

%%%%%%%%%%%%%%%%%%%%%%%%%%%%%%%%%%%%%%%%%%%%%%%%%%%%%
% Remarques et améliorations :
%%%%%%%%%%%%%%%%%%%%%%%%%%%%%%%%%%%%%%%%%%%%%%%%%%%%%
\section{Remarques et améliorations  :}

\subsection{Remarques sur le code}
TODO Remarques poids mémoire de la copie orginale des Images.

\subsection{Remarques sur les librairies Android}
TODO Remarques sur l'utilisation obligatoire d'une activité contexte pour charger une Bitmap.

\subsection{Améliorations à court terme :}
TODO
\clearpage

%%%%%%%%%%%%%%%%%%%%%%%%%%%%%%%%%%%%%%%%%%%%%%%%%%%%%
% Gestion du projet :
%%%%%%%%%%%%%%%%%%%%%%%%%%%%%%%%%%%%%%%%%%%%%%%%%%%%%
\section{Gestion du projet:}

\subsection{Organisation générale :}
TODO

\subsection{Avis personnels :}
TODO
\clearpage

%%%%%%%%%%%%%%%%%%%%%%%%%%%%%%%%%%%%%%%%%%%%%%%%%%%%%
% Conclusion :
%%%%%%%%%%%%%%%%%%%%%%%%%%%%%%%%%%%%%%%%%%%%%%%%%%%%%
\section{Conclusion :}
TODO



%%%%%%%%%%%%%%%%%%%%%%%%%%%%%%%%%%%%%%%%%%%%%%%%%%%%%
% Annexes :
%%%%%%%%%%%%%%%%%%%%%%%%%%%%%%%%%%%%%%%%%%%%%%%%%%%%%
\section{Annexes :}

\textbf{Cahier des charges :}
\\
\url{https://dept-info.labri.fr/~vialard/ANDROID/cahierDesCharges.pdf}
\\

\textbf{Représentation de la couleur:}
\\
Système de couleur RGB:
\\
\url{https://fr.wikipedia.org/wiki/Rouge_vert_bleu}
\\
Système de couleur HSV ou HSB:
\\
\url{https://fr.wikipedia.org/wiki/Teinte_Saturation_Valeur}
\\


\textbf{Documentation Android :}
\\
Activity:
\\
\url{https://developer.android.com/reference/android/app/Activity}
\\
Bitmap:
\\
\url{https://developer.android.com/reference/android/graphics/Bitmap}
\url{https://developer.android.com/topic/performance/graphics}
\\
Gestion des dimensions des Bitmap:
\\
\url{https://developer.android.com/topic/performance/graphics/load-bitmap}
\\
RenderScript:
\\
\url{https://developer.android.com/guide/topics/renderscript/compute}
\\
Utilisations des API:
\\
\url{https://developer.android.com/about/dashboards}
\\



\end{document}